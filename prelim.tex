\documentclass{article}

\usepackage{caption,color,fancyvrb,subfig}
%\usepackage{code,algorithmic,algorithm}
%\usepackage{algpseudocode}
\usepackage{algorithm}
\usepackage[noend]{algpseudocode}
%\usepackage[]{algorithm2e}
%\usepackage{graphicx,epsfig}
\usepackage{amsmath,amsthm,amsfonts}
%\usepackage{url}
\usepackage{graphicx}
\usepackage{setspace}
\usepackage{verbatim}
\usepackage{datetime}
\usepackage{float}
\usepackage{booktabs} 
\usepackage{tablefootnote}
\usepackage{multirow} 
\usepackage{amsthm}
\usepackage{float}
\newfloat{algorithm}{t}{lop}
 \floatname{algorithm}{Algorithm}
\newtheorem{theorem}{Theorem}[section]
\newtheorem{lemma}[theorem]{Lemma}
\newtheorem{definition}[theorem]{Definition}

\usepackage[backend=bibtex8,style=numeric,url=false,isbn=false]{biblatex} 

\renewbibmacro{in:}{}

\addbibresource{clustering_tools.bib}
\addbibresource{metagenomics.bib}
\addbibresource{string.bib}
\addbibresource{extra.bib}


%\begin{comment}
\textwidth = 6.5 in
\textheight = 9 in
\oddsidemargin = 0.0 in
\evensidemargin = 0.0 in
\topmargin = 0.0 in
\headheight = 0.0 in
\headsep = 0.0 in
\parskip = 0.2in
\parindent = 0.0in
%\end{comment}

\def\inv{^{-1}}

\newcommand{\ignore}[1]{}
\newdateformat{mydate}{\monthname[\THEMONTH] \ordinaldate{\THEDAY}, \THEYEAR}

\begin{document} 




\begin{titlepage}
\begin{center}

\textsc{\huge \bfseries \sc{Algorithms for clustering highly conserved phylogenetic markers}}\\
%\textsc{\huge \bfseries \sc{Title Line 2}}\\[1.0cm]
 
\emph{A prospectus submitted in partial fulfillment of the degree of Doctor of
Philosophy}\\[5.5cm]

%\textsc{\large DRAFT}\\
%\emph{October 18, 2011}\\[2.0cm]

\emph{Preliminary Oral Examination for}\\
\textsc{\large Christopher Michael Hill}\\[2.0cm] % [4.0cm]
\emph{Advisor:} \\
\textsc{Dr. Mihai Pop}\\[.5cm]
\emph{Committee Members:}\\
\textsc{Dr. Atif Memon}\\
\textsc{Dr. H\'ector Corrada Bravo}\\[4.0cm]

{\bfseries Department of Computer Science}\\
{\bfseries University of Maryland, College Park, MD 20742}\\
{\bfseries \mydate\today}
\vfill

\end{center}
\end{titlepage}

\newpage
\thispagestyle{empty}
\mbox{}

\setcounter{page}{1}
\doublespacing
\begin{abstract}

Microbes play a large role in every aspect of our lives.
They aid in digestion of food, have been associated with different diseases, and even help identify which keyboard and mouse we use.
In the environment, these microbes are responsible for most of the biochemical cycles that allow life to exist on earth, such as the carbon and nitrogen cycles.
Despite their importance, these organisms are among the least understood on Earth due in part to the difficult nature of culturing them in a laboratory setting.
Fortunately, advances in sequencing technology allow us to capture a small, noisy snapshot of the microbial community.
In typical analyses, the short fragments produced from these sequencing technologies first need to be clustered into similar groups based on some definition of similarity.
%The vast amount of sequencing data has led to researchers utilizing heuristics to cluster this data.

In this prospectus, we will describe the current approaches and limitations for clustering biological sequences, which are represented electronically as strings of characters.
First, we will discuss multi-core algorithms that speedup the runtime of clustering.
Then, we will describe efficient string matching algorithms to quickly recruit sequences to potential centers.
By sorting highly similar sequences, we provide a novel way to reduce the amount of work done during cluster recruitment.
Lastly, we will discuss the problematic issue of sequences that can align equally well to multiple centers and present strategies for handling them.


\end{abstract}
\setstretch{.5}

\newpage
\tableofcontents

\newpage
\listoffigures
\listoftables
\listofalgorithms

\newpage
\doublespacing


%\section{Introduction}
\section{Introduction}

Microbes play a large role in every aspect of our lives.
They aid in our digestion of food\cite{gill_metagenomic_2006}, have been associated with different diseases\cite{qin_metagenome-wide_2012}, and even help identify which keyboard and mouse we use\cite{fierer_forensic_2010}.
In the environment, these microbes are responsible for the most of the biochemical cycles that allow life to exist on earth, such as the carbon and nitrogen cycle\cite{venter_environmental_2004}.
Despite their importance, these organisms are among the least understood on Earth due in part to the difficult nature of culturing them in a lab setting.

Advances in sequencing technology allow us to capture a snapshot of the microbial community.
Unfortunately, this snapshot only can provide us with a very small, noisy picture of the microbial community.
A common genome size of a bacteria is often several million base pairs (bps) long and serves as the blueprint for building that organism.
Due to limitations in sequencing technology, we are only able to produce small fragments (called \emph{reads}), a couple hundred base pairs in length, from the organism.
Depending on the specific sequencing technology used, these fragments can be drawn randomly from the underlying genomes or from a specified region.

Often times, we want to learn what species of bacteria are found in an environment and their relative abundances.
Relative bacterial abundances often vary in several orders of magnitude.
Using methods to draw reads randomly from the underlying genomes are very expensive if we want to find rare species in an environment.
Fortunately, researchers have developed primers which allow us to sequence a specific conserved gene-- called the 16s ribosomal RNA (or 16s rRNA) gene)-- found in a large number of bacteria.
This gene is useful for phylogenetic studies because it is highly conserved between different species of bacteria.
16s rRNA genes contain nine hypervariable regions (V1-V9) that accumulate mutations overtime, providing us with species-specific signatures used for identifying bacteria.

A typical sequencing project involves first selecting or designing a primer that starts in a conserved region of the 16s rRNA gene and then allows the sequencing to extend through a couple hypervariable regions.
Once the reads have been generated, we now need to determine what reads came from what bacteria.
Since most of the bacterial species we encountered have never been seen before (NEED CITATION), we need to cluster (or group) similar reads together into operational taxonomic units (OTUs) based on sequence similarity.

The problem of clustering is a widely-studied problem in computer science\cite{fasulo_analysis_1999} and serves as the basis of this prospectus.
In this prospectus, we will first outline the current approaches used to cluster highly similar sequences.
Then we will describe our improvements to these existing methods made by utilizing multiple processors and efficient string matching algorithms.
Finally, we will discuss future work.


%In order to examine
%For a highly abundant species of bacteria, a large portion of our reads will be generated 
%For a given species of bacteria, we will have multiple reads that are generated from the population

%This gene is called the 16s ribosomal RNA (or 16s rRNA) gene.


%Computing the pair-wise distance are expensive and impractical in practice.



%\begin{figure}
%\begin{center}
%\includegraphics[scale=1.0]{img/sample_figure}
%\end{center}
%\caption{Sample camption}
%\label{fig:sample_figure}
%\end{figure}

%\begin{figure}
%\begin{center}
%\begin{verbatim}
%/* Sample code snippet */
%\end{verbatim}
%\end{center}
%\caption{Code snippet caption}
%\label{fig:sample_code}
%\end{figure}


%\section{Related Work}
\section{Related Work}

The traditional approach for clustering 16S rRNA sequences has involved the use of a multiple sequence alignment (MSA) for all sequences.

Optimal multiple sequence alignment between a collection of sequences can be done with a dynamic programming algorithm.

Impractical for the large number of sequences.
454 technologies can produce millions of sequences.

Greedy methods can be used to iteratively build a rooted tree.
Then a specific cutoff can be given to split the tree into clusters.

Alternatives include building a distance matrix between each pair of sequences.  
Once the distance matrix is built, clustering is typically done via hierarchical methods.
Agglomerative methods involve a bottom-up approach, where each sequence starts as its own cluster then iteratively merged with other clusters.
Divisive methods, on the other hand, work from a top-down approach, where all sequences belong to a single cluster then are iteratively split into smaller clusters.

While hierarchical methods requires less work than the multiple sequence alignment, calculating the pairwise distances generally requires cubic work, making it impracticable for a large number of sequences.

An alternative approach for clustering sequences involves selecting a sequence to become the cluster center.
The center is then used to recruit the remaining sequences that fall within some given \emph{distance} threshold to the center.
The distance between two sequences can include edit distance (also called Levenshtein distance), k-mer distance, similarity, identity, ...

\subsection{Greedy clustering paradigm}

A commonly-used clustering paradigm for sequence clustering is to iteratively select a sequence to serve as a cluster center and then recruit all remaining sequences that fall within some given distance of the cluster center.
This process is repeated until no more sequences remain or the predetermined list of centers is exhausted.

\subsection{Cluster center selection}
There are three strategies used for selecting potential cluster centers.
In {\bf \emph{de novo}} clustering, centers are selected \emph{only} from the set of input sequences.  Strategies for selecting potential centers are described below.
In {\bf closed-reference} clustering, a list of predetermined centers is given, such as a collection of previously discovered OTUs\cite{desantis_greengenes_2006,quast_silva_2013}.
In {\bf open-reference} clustering, sequences are first recruited to a list of predetermined cluster centers.
Afterwards, the centers are chosen by the \emph{de novo} methods.

Selecting which sequence to use as the cluster center is a difficult problem.
One strategy is to select the remaining sequence with the longest length.
An intuitive argument is that a longer sequence would more likely recruit shorter sequences.
However, the more rigorous argument is that selecting the longest remaining sequence allows you certain mathematical guarantees when aligning and recruiting shorter sequences.
For example, given 3 sequences: \emph{A}, \emph{B}, and \emph{C}.  If the length of \emph{A} is less than \emph{B} and \emph{C} and we know the distance between \emph{A} and \emph{B} and \emph{A} \emph{C}, we can not say anything for certain about the distance between \emph{A} and \emph{C}.
We do not have any data about the overlapping regions between \emph{B} and \emph{C}.

{\bf INSERT FIGURE OF THIS.}

This center selection strategy is used by CD-HIT\cite{li_clustering_2001}, DNACLUST\cite{ghodsi_dnaclust:_2011}.

Frequency of k-mers.

\subsection{Sequence recruitment}

A key part of any clustering algorithm is how the distance between two objects is computed.
In the case of sequence clustering, we need to calculate the distance between two strings.
Commonly-used distance metrics include:

\begin{itemize}

  \item Edit (Levenshtein) distance
  \item K-mer
  \item Identity

\end{itemize}

\subsubsection{Edit distance}

The \emph{edit distance} between a text string $t = t_1 t_2 ... t_n$ and pattern string $p = p_1 p_2  ... p_m$ is the minimum number of differences between them such that one string can be transformed into the other.
A difference is one of the following:

\begin{enumerate}

  \item A character of the text corresponds to a different character of the pattern.
  \item A character of the text corresponds to no character (a gap) in the pattern.
  \item A character of the pattern corresponds to no character (a gap) in the text.

\end{enumerate}

Algorithms for $k-mismatches$ only satisfies differences of type 1.

We use the Needleman–Wunsch{\bf NEED CITATION} dynamic programming algorithm to calculate the edit distance between two strings.


 Algorithm to solve $k$-differences 

\begin{equation}
\begin{aligned}
M[i, j] = \quad min
\begin{cases}
\quad M[i - 1, j] + 1 \\
\quad M[i, j-1] + 1 \\
\quad M[i-1, j-1] + \begin{cases} 
0, \quad \text{if }t_i == p_j \\
1, \quad \text{else } \\
\end{cases}
\end{cases}
\end{aligned}
\end{equation}



% Edit distance computation
 \begin{algorithm}
 \caption{Compute Edit Distance between two strings.  $O(nm)$ work.}\label{edit_distance}
 \begin{algorithmic}[1]
 \Procedure{ComputeEditDistance}{$a,b$}
 \State $n\gets |a|$
 \State $m\gets |b|$

 \For{$i=0 .. n$}   \State $M(i,0) \gets i$ \EndFor
 \For{$i=0 .. m$}
  \State $M(0,i) \gets i$
\EndFor

 \For{$i=1 .. n$}
 \For{$j=1 .. m$}
  \State $row \gets M(i-1,j) + 1$ \Comment Number of edits with a gap inserted into $b$
  \State $col \gets M(i,j-1) + 1$ \Comment Number of edits with gap inserted into $a$
  \State $diag \gets M(i-1, j-1)$ \Comment Number of edits with matching characters $a_i$ and $b_j$
  \If{$a_i \neq b_j$}  $diag \gets diag +  1$ \EndIf
  \State $M(i,j) \gets \text{min}(row,col,diag)$
\EndFor
\EndFor
\Return $M(n,m)$
\EndProcedure
\end{algorithmic}
\end{algorithm}

If we want to find a global alignment within $k$-differences, we only need to worry about a $2k+1$ band along the diagonal.
This reduces the amount of work from $O(n^2)$ to $O(nk)$.
However, this assumes we are aligning the two sequences end-to-end.



%\section{Preliminary Work}
\section{Preliminary Work}

\subsection{Parallelizing sequence recruitment to a cluster center}

Due to the large scale of sequencing data produced, clustering tools must utilize multiple processors to process the data in a timely manner.

Here, we present two parallel approaches for recruiting sequences to a cluster center.
The first approach (na{\"i}ve) is based on evenly partitioning the sequences among the processors.
The second approach (work-based) involves partitioning the sequences based on the potential work that needs to be done when calculating the edit distance to the center.
If the sequences are stored in a trie-like data structure, then it is beneficial to partition highly similar sequences together despite potentially assigning an uneven number of sequences to each processor.

We implement these parallel approaches in DNACLUST\cite{ghodsi_dnaclust:_2011} show the speed-ups when clustering tens of millions of 16S rRNA sequences. 

\subsubsection{Na{\"i}ve parallelization strategy}

The second step of DNACLUST's algorithm involves recruiting all sequences that lie within a given distance of the current cluster center.
Given $p$ processors, we can evenly partition the database into $p$ chunks such that each processor can calculate edit distance independently in parallel.

{\bf INSERT FIGURE}

\subsubsection{Work-based parallelization strategy}

Since we reuse part of the dynamic programming matrix, evenly partitioning the sequences may split highly similar sequences into separate threads.

Instead of evenly splitting the number of sequences between threads, we can evenly split the amount of potential work (characters we need to examine in the trie).

This is done by counting the total number of characters on the edges in the trie (trie length) and dividing by the number of threads.

{\bf INSERT FIGURE}


\subsection{Efficient data structures for edit distance computation}

Currently, we require $O(n^2)$ work to calculate the edit distance between two sequences.
This cost is reduced to $O(nk)$ in the specific case of globally aligning two sequences within $k$ edits.

In this section, we describe how how to further improve the runtime to $O(k^2)$ in the case of global alignment and $O(nk)$ for semi-global alignment.

\subsubsection{Alternative representation of the dynamic programming matrix}
When calculating the edit distance between sequences $A = a_1 a_2 .. a_n$ and $B = b_1 b_2 .. b_m$, entry ($i,j$) in matrix $M$ represents the minimum edit distance between prefixes $A_{1,i}$ and $B_{1,j}$ (Algorithm \ref{edit_distance}).

An alternative way to view this alignment is to consider each diagonal $d$ and edit $e$ of $M$.  The $d$-diagonal is equal to $i-j$.
Let $C$ be another matrix where entry ($i,j$) now refers to the furthest reaching row in $M$ of diagonal $d$ that contains $e$ edits.

-Show that the matrix solves the problem.

-Show that we can write a recurrence to solve this problem.

-Show the algorithm pseudocode.

-LCP can be answered in constant time via a suffix tree (gusfield).

-Compare the two approaches in terms of cells of the dp matrix that need to be computed.

-More difficult to exploit the sorted sequences (possible future work).

-Future work includes actually implementing the O(1) LCP extension.

\subsubsection{Suffix tree cluster center representation}

\subsection{Handling ambiguous reads}

When a sequence is being recruited by a center, it is possible that this sequence is within some distance from another potential center.
Henceforth, we refer to a sequence that lies within a given distance from multiple centers as \emph{ambiguous}.
Currently in DNACLUST, an ambiguous sequence is recruited by the first center that encounters.
Depending on the number of ambiguous of sequences, this may affect the resulting cluster abundances.
Furthermore, downstream analyses on analyzing these count matrices (such as detecting differentially abundant OTUs) could lead to incorrect results.

Here, we describe different methods for assigning ambiguous reads.

The first way is to simply discard any ambiguous reads and only consider reads that can be uniquely aligned to a single center.

Another way is to randomly assign the ambiguous read to the set of potential centers.

Similarly, instead of randomly assigning the reads, we can assign a fractional count to each center.

Lastly, we can assign a read based on the proportion of uniquely aligned reads to the center.  In other words, if a read can align equally well to two different centers, but one center contains uniquely aligned reads and the other contains none, then it is more probable that the read came from the first center.

{\bf INSERT FIGURE}


%\section{Proposed Work}
%\section{Proposed Work}

\subsection{Farrah's algorithm for SIMD edit distance computation}

\subsection{Streaming clustering}


\section{Timeline}

\begin{center}
\begin{tabular}{|l|c|}
\hline
Parallel DNACLUST & 1 months \\
Fast, exact clustering tool using efficient $k$-diff algorithm & 2-4 months \\
Handling ambiguous read and their affect on differential abundance & 1 months \\
\hline
\textbf{TOTAL} & 4-6 months \\
\hline
\end{tabular}
\end{center}

Paper deadline goals:

\begin{itemize}
\item 14th Workshop on Algorithms in Bioinformatics (WABI), May 2014: Parallel DNACLUST and ambiguous read assignment strategy
\item Nucleic Acids Research, September 2014: Efficient k-difference clustering tool
\end{itemize}


\section{Conclusion}

Microorganisms play an important role in the environment.
Despite this, these microbes are among the least understood on Earth due in part to difficulties culturing them in a lab setting.
Sequencing technology provides us with a solution, allowing us to get a snapshot of the microscopic communities that inhabit these environments.
Unfortunately, this snapshot only can provide us with a very small, noisy picture of the community.
In typical analyses, the short fragments of DNA produced from sequencing technologies need to be clustered into similar groups based on some similarity.
In this prospectus, we have described the current approaches and limitations for clustering highly similar DNA sequences.
By leveraging efficient string matching algorithms and multi-core systems, we have proposed a fast and exact clustering tool for DNA sequences.

\newpage


\section{Acknowledgments}

I owe my sincere gratitude to all the people who made this work possible.
I'd like to thank my adviser Mihai Pop for giving me the opportunity to work on intellectually challenging and interesting projects for nearly 8 years.
His guidance and support has led me to pursue a career in academia and I am forever grateful.
Furthermore, I would like to thank my committee members and mentors Atif Memon and H\'ector Corrada Bravo for their support.
I owe a great thanks to all the members of Pop, Bravo, and Memon labs for their discussions, contribution, and friendship.

\appendix
\section{Reading List}
\label{app:readinglist}

\subsection{Clustering}

%\addbibresource{clustering_tools.bib}
\printbibliography[keyword=chriscluster,heading=subbibliography,heading=none]

\subsection{String algorithms}

\printbibliography[keyword=chrisstring,heading=subbibliography,heading=none]

\subsection{Metagenomics}

\printbibliography[keyword=chrismetagenomics,heading=subbibliography,heading=none]


\newpage

\nocite{*}
%\printbibliography

%\bibliographystyle{plain} 
%\bibliography{prelim}


\end{document}

